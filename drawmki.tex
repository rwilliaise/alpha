
To facilitate the management of the window and drawing context, Draw: Mark I was
the first iteration of the Draw component.

\section{Purpose}

The Draw module has to achieve and manage several different tasks.

\begin{enumerate}
    \item Window creation and deletion.
    \item OpenGL context creation and deletion.
    \item Rendering objects.
    \item Main window interaction loop.
\end{enumerate}

To achieve window creation and deletion, GLFW is used. This ensures reliability
as GLFW is a well-tested and established library for window and context
creation. To manage the context, a generated version of glad is used.

\section{Usage}

The primary arbiter to \texttt{drawmki} is the \texttt{drawmki::window} class.
This handles the window creation as well as GL context management.

\section{Tests}

To ensure that rendering is fully reliable and in working order, a multitude of
tests were created. These can be seen in Figure \ref{fig:drawmki/tests}.

\begin{figure}[h]
    \centering
    \bgroup
    \def\arraystretch{1.5}
    \begin{tabular}{| l | l | p{.25\textwidth} |}
        \hline
        & Dependencies & Purpose \\ \hline
        \texttt{window00} & \texttt{drawmki} & To demonstrate window creation
        and main window loop. \\ \hline
    \end{tabular}
    \egroup
    \caption{Draw: Mark I tests and purposes}
    \label{fig:drawmki/tests}
\end{figure}

