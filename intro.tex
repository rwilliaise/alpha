A core part of engineering, especially modern engineering, is the communication
of ideas to others, whether it be for designers to manufacturers, or even
designers to designers. Software, like hardware, is to be designed and
engineered, and most of all, manufactured as in writing it. The purpose of this
project is to demonstrate the importance of communication in modern software
engineering, and create an application second.

The primary inspiration behind Alpha was the Apollo Space Program initiated
under the Kennedy Administration. Not only did the NASA Manned Spacecraft Center
team have time pressure, the nature of the program required an intense testing
protocol, to ensure the safety and security of the astronauts. This required a
high level of communication, requiring contractors and subcontractors to test to
the specific guidelines laid out by the many engineers and boards assembled at
NASA.

This pressure found in the Apollo program is reflected in how Alpha's design and
mission is guided. Before any of the software was designed, principles guiding
design were created. As well, each component, when fully redesigned or otherwise
changed in an interface-breaking way, will be designated as a new "model" or
"mark" of the component. As an example, the first version of the Draw component
is called "Draw: Mark I."

\section{Principles}

To achieve a high quality final product, principles must be laid out to guide
the creation of both hardware software. These principles borrow a lot from the
hardware design principles during the Apollo space program, as written in NASA
SP287.

\begin{enumerate}
    \item Build off of established software.
    \item Minimize interfaces between components.
    \item Make each component reusable.
    \item Small steps go a long way.
    \item Use as much experience as possible from each step.
    \item Make each component reliable.
\end{enumerate}

Using established software ensures reliability and, in some cases, performance
of a component. This allows for an increased focus on the more difficult and
less tedious portions of software design, which is integral for a team on a time
crunch or a team without sizable manpower or wherewithal. Established software
is also easy to come by in the form of free, open-source software.

Minimization of interfaces and making components reusable go hand in hand,
especially in software. By minimizing the interfaces between components, it is
inherently easier to pull a component out of it's codebase and use it inside of
another codebase. Another part of this is ensuring that each component has
isolated build-system files. As in each build-system is separated and can build
freestanding without needing to be inside the Alpha codebase.

\section{Mission Statement}

The purpose of the Alpha program is to simulate and display a possible rocket
launch from Earth to Mars. The required software to achieve this are as follows:

\begin{enumerate}
    \item \label{itm:intro/n_body} A $n$-body simulation of the solar system.
    \item \label{itm:intro/trajectory} Method to calculate the trajectory required to reach Mars by rocket.
    \item \label{itm:intro/rocket} Simulation of rocket thrust and the subsequent decrease of mass.
    \item \label{itm:intro/window} Method to open and manage the window and drawing context.
    \item \label{itm:intro/renderer} Renderer of the $n$-body simulation.
\end{enumerate}

For Item \ref{itm:intro/n_body}, to maximize the re-usability of this component
in future projects, this piece will be it's own component, codenamed $n$-body.
In the source code of Alpha, this component can be found under the
\texttt{nbody} folders or repositories.

For Items \ref{itm:intro/trajectory} and \ref{itm:intro/rocket}, as these are
closely related, these two pieces will be combined into their own component,
codenamed Rocket. In the source code of Alpha, this component can be found under
the \texttt{rocket} folders or repositories.

For Item \ref{itm:intro/window}, to maximize the re-usability of this component
in future projects, will be it's own component, codenamed Draw. In the source
code of Alpha, this component can be found under the \texttt{draw} folders or
repositories.

For the final Item \ref{itm:intro/renderer}, this piece is closely related to
however not fully required for the $n$-body simulation. As a result, it will
need to be a separate component to ensure that a dependency on Draw for $n$-body
is not required. This component will be called the $n$-body Renderer. In the
source code of Alpha, this component can be found under the
\texttt{nbodyrenderer} folders or repositories.

A majority of these components require a substantial amount of linear algebra.
To build off of established software and maximize reliability of all components,
glm will be used as a dependency for all of the components listed.

